\documentclass[a4paper,12pt]{article}
\usepackage{anysize}
\marginsize{3cm}{3cm}{2.5cm}{2.5cm}
\title{A Definitive Notation for Electronic Analogue Computing}
\date{October 04}
\author{Charles Care}

\newcommand{\code}[1]{\hspace{0pt}\hbox{\texttt{#1}}}
\renewcommand{\th}{$^{th}$}

\newsavebox{\myquoteEnd}
\newenvironment{myquote}[1]
{\begin{sloppypar}\begin{quote}\small\sbox{\myquoteEnd}{#1}\itshape}
{ \ \\ \hspace*{\fill} \usebox{\myquoteEnd} \end{quote}\end{sloppypar}}

\newenvironment{fragment}{\begin{quote}\small}{\end{quote}}


\newenvironment{centrefigure}
{\begin{figure}[h] \begin{center} }{\end{center}\end{figure}}


\begin{document}


\maketitle


This documentation refers to both the \%analog 
notation, and the basic engine that comes with it.

Together these two components comprise version 1.0 of the \%analog notation. Version 1.0 couldbe seen very much as an initial attempt at producing a tool to enable analogue modelling. It could be imagined that what is required is something more broad than simply a notationfor electronic analogue computers but this is certainly a start.

In its current state the models produced with this notation are unfortunately rather digital in certain behaviour. These issues will be identified and discussed in the `where from here' section.

\section{An introduction to the electronic analogue computer}

The electronic analogue computer was a...

\subsection{Principle components}
An analogue computer consisted of a number of modules. Each module consisted of an operational amplifyer that could be used as either an integrator or a summer. Typically each module also contained two potentiometers.

\subsubsection{The Potentiometer}
