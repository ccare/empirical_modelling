\documentclass[a4paper]{article}

\usepackage{mathrsfs}
\usepackage{alltt}

\usepackage{anysize}
\marginsize{3cm}{3cm}{2.5cm}{2.5cm}

\newsavebox{\savepar}
\newenvironment{boxcode}{\begin{lrbox}{\savepar}
	\begin{minipage}[b]{\textwidth}\begin{alltt}}
{\end{alltt}\end{minipage}\end{lrbox}\fbox{\usebox{\savepar}}\ \\}


\title{\%physical \\ \itshape \large
   A definitive notation for modelling \\with physical quantities}
\author{Charlie Care}
\date{January 2005}

\begin{document}
\maketitle
\section{Introduction}

It was in a second-year revision lecture that I came across a formula that did 
not make sense. I was attempting to understand the subject matter and found 
myself puzzling over it. I tried putting some numbers in and 
there I found the fault. The offending equation was not homogeneous.

My state of confusion lasted for a week or two until I found an old exam 
question that provided the same equation for a slightly different scenario. The 
difference between the two scenarios came down to one constant in the equation. 
In the original equation this constant had a value of 1 and in the new 
equation, a value of 2. Since it was just a multiplication by unity, the 
constant had been left out of the first equation, but there was the problem. 
The 
mysterious constant of value 1 had a dimension. Although it was a numerically 
unnecessary feature of the first equation, it was a metrologically essential 
component for the mathematics to make sense. The process of checking the 
homogeneity of an equation is commonplace in the Physics classroom -- students 
are trained to use the SI unit derivations of the various quantities in an 
equation and to compare the dimension of the right hand side with the left. If 
the two dimensions are not equivalent, then the equation is certainly 
incorrect. If the two dimensions \emph{commensurate}\footnote{This term is used 
in the 
majority of the referenced literature to indicate the compatibility of two 
dimensions.} then the equation might be correct. Checking dimensionality can 
expose errors but is not a proof technique in itself.

Formality in the structure of computing languages is often argued to be 
necessary if software is ever to be a robust product. In functional languages 
such as SML, there is great emphasis placed on the importance of \emph{type}. 
In conventional programming languages, variables are given types such as 
string, integer or real. Typing variables links in with the representation of 
quantities on our machines and had great importance when computer memories were 
small. The distinction between integer and real is however quite unnecessary 
for the programmer who is uninterested in the 
internal representation of data.\footnote{Much thought has gone into the 
relationship between \emph{type}
and \emph{dimension}. Gehani \cite{ME_gehani} felt that ``units of measure 
reflect the physical characteristics of a quantity while the type indicates how 
the magnitude of the quantity being measured is to be stored.''} 

In a strongly typed language such as SML, there are particularly strict 
conventions relating to the use of types. For example, in SML there is no 
direct provision to add an integer to a real; the statement {{\texttt{a = 
1.0+2}}} is meaningless unless the programmer provides a definition of what the 
\texttt{+} operator should do when supplied with a real number and an integer. 
Restrictions of this kind are claimed to be sensible because type checking 
provides a mechanism to check that the program represents something 
meaningful. Getting the types correct is thought to be a good indication of 
sensible semantics.

I would argue that types are almost unnecessary in a programming 
language\footnote{I came to this conclusion during a conversation with Prof. 
Eric Roberts (Stanford University). His view was that although type checking 
may eliminate a number of  errors, these are usually the errors that even a 
novice programmer would notice immediately. The tkeden tool has very limited 
type constraints -- this is a pattern being followed by modern `scripting' 
languages (Perl, Python etc.). In the 
literature relating to implementing variables with dimension, attempting to mix 
both units and types seems to
cause unnecessary complexity.} and that the programmer should be liberated from 
having to define internal representations of data. I say almost because it is 
useful to think of strings as separate from numbers. For the rest of this paper 
I will be concerned with data that represents numbers.
These will be mainly numerals, but I also will use characters as algebraic 
variables -- where the strings simply represent data but are not actually data.

Removing the need for types will enable those who work with computers to be 
able to think about numbers in the way that is more in tune with the way we 
think about quantities in everyday situations. 

\section{Dimensions and programming}

The idea of using dimension as a error-checking mechanism for programming 
has been suggested by a number of people during the last forty years. 
Most writing on the subject follows the lead of \cite{ME_karrLoveman} and 
claims that the earliest mention of the idea was by Cheatham.\footnote{The 
paper references: Cheatham, T. Handling fractions and n-tuples in algebraic 
languages. Presented at the 15th ACM Annual Meeting, Aug,1960.} 
This work provides dimension as an augmentation to the 
existing type-systems: ``Within a restricted application area one can go 
further and categorise objects not just by their type but also by their 
meaning'' \cite{ME_trg_progCog}. Early work saw the role of units as 
``partitioning
what would normally be the mode REAL'' \cite{ME_karrLoveman}.

\cite{ME_grosu} and \cite{ME_hall} discuss classic software problems like the 
crash of NASA's Mars Climate Orbiter
 due to numerical values that represented imperial quantities being 
interpreted as metric quantities. They claim that a programming language that
supported units of measure would have had been able to protect against
such errors.
Although the application of dimension has a use in formal verification of 
software \cite{ME_grosu,ME_hayes} 
possibly more important is the 
``increased readability'' pointed out by \cite{ME_karrLoveman}. Attaching extra
information to a programs variables can make the code more accessible to a 
future reader.

\section{Dimensionality of observables in Empirical Modelling} 

I think that Empirical Modelling (EM) is an ideal environment
 to explore the power of programming with dimension. Firstly, the discipline 
of EM makes great use of observation as a concept;
preferring to refer to a model's observables rather than a program's variables.
Secondly, tkeden (the EM tool) is particularly well-suited to the construction
of domain-specific programming notations. By using the tkeden notation 
framework, it should be possible to enhance parts of existing models with the 
new notation (see section \ref{vccs}). Comparing the readability of a model 
with and without the assistance of dimensional notation will indicate how 
successful such notation might be.

Observation relates the visible state of a model to experience. By this we
mean that observables directly relate to the referent of a model and convey a 
meaning richer than a value.
Dimension units 
might be found to provide some of that context and help enhance tkeden's 
support for observable-oriented modelling. Currently, the context of an 
observable is only given by its name and its
importance in the modeller's mind. The philosophy of EM has been compared to 
James' Radical Empiricism with the dualism
of mental model and computer model coming together into one computer-based 
\emph{construal}.\footnote{
Need a reference for this.} One difficulty with this
position is that it relies heavily on the assumption that the meaning that a 
modeller builds into his or her model can
be communicated to others through interaction and experience of the model. Due 
to the complexity of software, EM artifacts are not always straightforward to 
understand and therefore meaning attached to the model is never communicated
from the one modeller to the next. It is a hope that adding this extra 
information to the observables will help expose the meaning that a modeller is 
attaching to an observable in their modelling.

\section{\%physical}
Constructing a unit system for eden is fairly straight-forward if the
agent-oriented parser\footnote{The agent-oriented parser (aop) is a parser 
built into the tkeden
tool. It is particularly useful for building new notations because the rules 
that govern
parsing are stored as internal eden data structures. The flexibility of being 
able to 
dynamically modify such rules supports on-the-fly experimentation with the 
structure of a 
programming notation.} is used.
 This allows the creation of a new \emph{notation} 
which can be given 
any textual structure required. The notation will be called \%physical -- the 
name \emph{Physical} was also 
given to an early prototype implementation of dimension using Pascal 
\cite{ME_physical}.

Previous work in the field has used a vector of integers, reals or bits to
represent different exponents \cite{ME_baldwin,ME_cleaveland,ME_hamilton}. I 
have followed this implementation strategy, but  since
no significant distinction is made in eden between the numeric types real and 
integer,
 I have used a list of numbers.

This list of seven numbers represents the exponents of the seven SI base 
dimensions. The current implementation of \%physical only supports these seven 
base dimensions;
 adding new dimensions will be discussed later.

The previous work has always been done in the compiler, or in a 
pre-processor to the compiler \cite{ME_physical}.
 This meant that dimension checking was done at compile-time and there 
 have always been discussions 
about whether dimension information should be available at run time. An 
important 
point was raised in the literature regarding whether dimension checks should 
be done in the compiler which would check for all such errors.
 It was thought by []
that there was no use in having
run-time code -- possibly in a safety-critical application  -- throwing an error
because a length was added to a mass. 

Because eden is an interpreted language these issues are not hugely relevant.
The main motivation for providing unit support in eden is to provide the 
modeller with a more elaborate toolkit of notations that allow him or her to 
think about their modelling in a more efficient way. It is hoped that physical
units would bring the discussion of observables in the modelling process closer 
to the referent being modelled.

However, there is also an essence of the program (or model) reliability issue 
attached
to this work. Bringing the modeller closer to the physical
structure of observables will not only aid the modelling process, but also 
assist in
making models more accessible and easier to understand. This might help 
modellers reason about the reliability of their models more 
effectively.\footnote{
\%physical could be adapted to support 
a safety-critical system. In \%physical, values that are dimensionally 
inconsistant are
represented as undefined. Such a situation could be protected against if the
dependency maintainer was augmented 
so that the modeller would be asked to confirm any input 
that results in a value becoming undefined.}

As referred to above, the implementation uses a list of seven numbers to 
represent the dimension of an observable. The elements of this \emph{dimension
vector} refer to the exponents of the different base dimensions in the 
following order:
length, mass, time, electric current, temperature,
amount of substance, luminous intensity. Hence the list
 \texttt{[1,0,-1,0,0,0,0]} represents 
length\textbullet time\raisebox{1ex}{\small -1} -- length per unit time or 
more simply \emph{speed}.

Each dimension has a base unit, these are (respectively): metre (m),
kilogramme (kg), second (s), Ampere (A), Kelvin (K), Mole (mol) and candela 
(cd).

As well as these standard units, there are standard prefixes to allow for 
scaling. For example, the length 12mm (twelve millimetres) 
corresponds to $12\times10^{-3}$m and the kilometre (km) refers to one 
thousand metres. The prefixes have not been implemented in the initial 
implementation due to a number of issues:

\begin{enumerate}
\item Consider the quantity \texttt{12 mmol}: is this 12 millimoles 
($12\times10^{-3}$ mol), 
or is it 12 metre\textbullet moles? There is no guarantee that the prefixes 
have their own name-space.
\item The base unit for mass -- kg -- has a standard prefix built into it. 
While ms corresponds to
a scaling factor of $10^{-3}$ of the base unit of time, mg corresponds to 
$10^{-6}$ scaling of the base unit for mass.
\item Some of the standard prefixes are not ascii characters (e.g. $\mu$) so 
the set of prefixes cannot
be completely represented by a single character. 
\end{enumerate}

The literature has not gone very far in looking at how these prefixes could be 
included in such a notation.
M\"{a}nner \cite{ME_manner} saw units with prefixes as another derived unit.

\begin{alltt}
UNIT g;        (* mass    *)
     cm;       (* length  *)
     sec;      (* time    *)
     cents;    (* money   *)
     \hfill\cite[page 14]{ME_manner}
\end{alltt}

From defined units it is also possible to declare a unit as a derivation.

\begin{alltt}
UNIT k  = 1000;                  (* scale factor    *)
     kg = k * g;                 (* mass            *)
     m  = 100 * cm;              (* length          *)
     N  = kg * m/(sec * sec);    (* force           *)
     \hfill\cite[page 14]{ME_manner}
\end{alltt}

This is not a very elegant solution because every possible scaled unit would 
have to be manually defined.
A number of other writers criticise M\"anner for this position but it's not 
clear how they plan to get around the problem.
	
\section{An example \%physical translation \label{translations}}

The notation \%physical will accept physical quantities as input. However, the
parser will need to translate the code into eden variables and dependencies.
In this section, we will look at some simple \%physical statements and show (as 
boxed code)
what the parser translates these statements into.

\begin{alltt}
%physical
A = 5 kg;     ## Example of %physical
B = 10 kg;    ##  assignment
\end{alltt}
\begin{boxcode}
%eden
PH_val_A is 5;                 ## Eden representation of the value of 5 kg
PH_dim_A is [0,1,0,0,0,0,0];   ## Eden representation of the dimension of 5 kg

PH_val_B is 10;                ## Eden representation of the value of 10 kg    
PH_dim_B is [0,1,0,0,0,0,0];   ## Eden representation of the dimension of 10 kg
\end{boxcode}

The sum of two quantities is defined if and only if they have the same 
dimension.
This is managed in the translated eden code below by adding a guard to the 
\texttt{PH\_val\_C} formula that checks whether the associated dimension is
defined.

\begin{alltt}
%physical
C=A+B;
\end{alltt}
\begin{boxcode}
%eden
PH_val_C is (PH_dim_C != @) ? PH_val_A+PH_val_B : @;
PH_dim_C is (PH_dim_A == PH_dim_B) ? PH_dim_A : @;
\end{boxcode}

Difference is defined in the same way.

\begin{alltt}
%physical
D=A-B;
\end{alltt}
\begin{boxcode}
%eden
PH_val_D is (PH_dim_D != @) ? PH_val_A-PH_val_B : @;
PH_dim_D is (PH_dim_A == PH_dim_B) ? PH_dim_A : @;
\end{boxcode}


The product of two quantities is defined for all dimensions.
The dimension of the result is the sum of dimension vectors for the two
quantities. An eden function \texttt{PH\_vectorAdd} is provided that performs 
pairwise addition on two lists of equal length.
\begin{alltt}
%physical
E=A*B;
\end{alltt}
\begin{boxcode}
%eden
PH_val_E is PH_val
_A*PH_val_B;
PH_dim_E is PH_vectorAdd(PH_dim_A,PH_dim_B);
\end{boxcode}

The quotient of two quantities is defined for all dimensions.
The dimension of the result is the difference of the operands dimension 
vectors. \texttt{PH\_vectorMinus} performs pairwise subtraction on the 
different exponents.

\begin{alltt}
%physical
E=A/B;
\end{alltt}
\begin{boxcode}
%eden
PH_val_E is PH_val_A/PH_val_B;
PH_dim_E is PH_vectorMinus(PH_dim_A,PH_dim_B);
\end{boxcode}

\section{Implementation issues}
In the final implementation, it was more straightforward to put the 
dimension guards introduced in the previous section
 on all values rather than just for formulae involving addition
and subtraction. This modification made translation slightly easier because the 
right hand side 
of a definition could be translated independently of the left hand 
side.\footnote{
If guards were added by the rules matching the \texttt{+} and \texttt{-} 
operators
then these rules would have to know what the left-hand side of the definition 
was (because the 
guard refers to a dimension corresponding to the left-hand side). By giving all 
definitions a guard,
the code of the guard can be generated by the same parser rule that parses the 
left hand side.}

Also it was found necessary to re-write the predicate used in the definition of 
a
sums (or differences) dimension to a function names \texttt{PH\_dimensionComp}. 
This was because not all statements 
consisted of one operator and two operands. For example, consider how the
following might be translated:

\begin{alltt}
%physical
A = b+c+d;
\end{alltt}
\begin{boxcode}
%eden
PH_val_A is (PH_dim_A != @) ? b + c + d : @;
PH_dim_A is (PH_dim_b == PH_dim_c == PH_dim_d) ? PH_dim_b : @;
\end{boxcode}

Here the predicate would translate to a three-term equality test which is not
compatible with the eden syntax.\footnote{It's not even enough to add 
parentheses. E.g.
\texttt{(PH\_dim\_b == (PH\_dim\_c == PH\_dim\_d))}, because the predicate 
\texttt{(PH\_dim\_c == PH\_dim\_d)}
just returns true or false and so cannot be matched with \texttt{PH\_dim\_b}.}
 The solution was to replace the test with
\texttt{PH\_dimensionComp}. This function takes two dimensions and returns 
undefined (denoted by the symbol \texttt{@}) if they do not match. If however 
they do match then the 
function returns the dimension. This provides a nested-comparison solution to
the above problem and also allows for sub-expressions in parentheses to be 
added to \%physical notation.

\begin{alltt}
%physical
A = b+c+d;
\end{alltt}
\begin{boxcode}
%eden
PH_val_A is (PH_dim_A != @) ? 1 + 2 + 3 : @;
PH_dim_A is  PH_dimensionComp(PH_dim_b,PH_dimensionComp(PH_dim_c,PH_dim_d));
\end{boxcode}

\section{Description of the syntax of \%physical}

There are two types of statement in the \%physical notation. The first is
\emph{querying} the value of \%physical observables and the second is 
\emph{definition}. Querying in the \%physical notation 
uses the same syntax as in eden and scout. An observable name is
prefixed with the \texttt{?} symbol and information about that observable
is printed to the terminal. Other than this construction, every other
statement is a form of definition and uses the \texttt{=} symbol.

\subsection{Querying}
The provision for being able to query observables relates closer to 
the experimental nature of EM. It allows a human agent, to observe the
state of the model and to follow dependencies. As described above, querying is
denoted with the \texttt{?} operator and it has the following syntax.

\begin{alltt}
? <NAME> ;
\end{alltt}

The syntax of \texttt{<NAME>} will be defined in the next section.

\subsection{Definition}

A definition in \%physical creates translated
eden dependencies. The syntax of a definition (or redefinition)
has the following form:

\begin{alltt}
<NAME> = <EXPRESSION> | <FUNCTION> | <FUNCTION_DEFINITION> ;
\end{alltt}

\paragraph{Names}
A \%physical name follows standard variable name conventions. Any combination
of alpha-numeric characters (upper and lower cases) and the underscore 
(\texttt{\_}) is allowed, providing that the first character of a name is not a 
numeric digit or an underscore.

The \%physical name-space uses a subset of the eden name-space. A \%physical
name maps to two eden observables, one of these represents the \emph{value}
of the  \%physical name and the other the \emph{dimension}.\footnote{This was 
seen in the example
translations given in section \ref{translations}.} 
For the \%physical name \texttt{myName}, the corresponding eden observables
are \texttt{PH\_val\_myName} for the value and \texttt{PH\_dim\_myName} for the
dimension.

As a regular expression, the syntax for \textbf{name} is:

\begin{alltt}
<NAME> = [a-zA-Z][a-zA-Z0-9_]*
\end{alltt}

\paragraph{Expressions}
In the simplest case an expression can just be another \%physical name or a 
literal value. Examples of these are the following \%physical definitions.

\begin{alltt}
%physical
a = b;       ## Dependency between two %physical observables
b = 12 m;    ## Defining b to be the literal quantity, 12 metres.
\end{alltt}

However, expressions can also be the formula constructed with the operators
\texttt{+},\texttt{-},\texttt{*},\texttt{/}. Parentheses can also be used to 
represent sub-expressions.

\begin{alltt}
<EXPRESSION> = <NAME> | <LITERAL> | <FORMULA> | ( <EXPRESSION> ) 
<FORMULA>    = <EXPRESSION> + <EXPRESSION>
             | <EXPRESSION> - <EXPRESSION>
             | <EXPRESSION> * <EXPRESSION>
             | <EXPRESSION> / <EXPRESSION>
\end{alltt}

For example, other valid \%physical definitions are:

\begin{alltt}
%physical
a = b*c;
b = (b+c)*(e/(f-j));
\end{alltt}

\paragraph{Literals}
In \%physical, there are three forms of literals. The first two are numeric 
quantities and either have a dimension or are dimensionless.
The third allows a value to be linked to an eden observable. For example:

\begin{alltt}
a = 30 m;      ## 30 metres, a length
b = 23 kg;     ## 23 kilogrammes, a mass
c = 12 ms\{-1\}; ## 12 metres-per-second, a speed
d = 11;        ## 11, a dimensionless quantity
e = eden\{a\};   ## The value of the eden observable "a"
\end{alltt}

Dimensions are given with their base units. These are the seven SI units. 
Each unit can be given an optional exponent in curly-braces. If no exponent 
is given then it is taken as 1. If there is no occurrence of a particular
unit in a literal, then the exponent is taken as 0.
The syntax of a literal is:

{{{
\begin{alltt}
<LITERAL>   = <REAL> <UNIT> | eden\{<EDEN_NAME>\}
<REAL>      = [0-9]+(.[0-9])*
<EDEN_NAME> = [a-zA-Z_\-][a-zA-Z0-9_\-]* 

<UNIT>      = [<M>][<KG>][<S>][<A>][<K>][<MOL>][<CD>]
<M>         = m   [<EXPONENT>]
<KG>        = kg  [<EXPONENT>]
<S>         = s   [<EXPONENT>]
<A>         = A   [<EXPONENT>]
<K>         = K   [<EXPONENT>]
<MOL>       = mol [<EXPONENT>]
<CD>        = cd  [<EXPONENT>]
<EXPONENT>  = \{ <NUMBER> \}
<NUMBER>    = [0-9]+
\end{alltt}
}}}

\paragraph{Function}
Just as in eden, functions can be on the right-hand side of a dependency.
Only functions defined in the \%physical notation 
can be called within \%physical.
Functions are used in the standard way:

\begin{alltt}
%physical
a = 30 m;
b = square(a);
\end{alltt}

The syntax is:
\begin{alltt}
<FUNCTION>  = <NAME> ( <NAME_LIST> )
<NAME_LIST> = <NAME> [, <NAME_LIST>]
\end{alltt}

\paragraph{Function definition}
Functions can be defined in \%physical. Inside a function, the programmer can
draw on the power of the eden language. The framework provided for defining
functions should allow powerful constructions to be added to the notation.

Inside a function a number of predefined features are available. Firstly, if a
parameter \texttt{a} has been defined then it is possible to refer to 
that parameters value and dimension using \texttt{a\_val} and 
\texttt{a\_dim} respectively. Secondly, a number of special functions are 
available to be used. These provide for multiplication and division of 
dimension.




For example, to produce a `square' function it will be necessary to take one
parameter and to square both its value and dimension. This is done with the 
following definition. Note the semi-colon after the last curly-brace, the whole 
block is one physical statement and therefore must end with a semi-colon.

\begin{alltt}
%physical
square = function (a) -> b \{
	## This is eden code below
	b_val = a_val*a_val;               ## Scalar multiplication

	b_dim = multDim(a_dim,a_dim);      ## Pre-defined function to allow
	                                   ##   dimensions to be multiplied
	                                   ##   in this 'eden' context 
\};
\end{alltt}
It must be pointed out that since the body of the function is eden code,
 the \texttt{*} operator in the last example is the
eden multiply operator and not the \%physical one. It therefore 
can only multiply scalar values and the dimensions have to be managed in a 
different line of code.
This may appear a little cumbersome; such a simple function
could be expressed in the notation as \texttt{a * a}. However, the 
intention is that this construction should provide for the situation where the 
user would
like to write functions that do not obey the standard dimensional algebra.

For example, it might be desirable to write a function that adds 15 to the
value of an observable. If the dimension of the observable is not known then 
it is difficult to write down such a generic formula. However, the function
construction provides enough freedom from the constraints of the notation to
do exactly what's needed.

\begin{alltt}
%physical
add15 = function (a) -> b \{
        b_val = a_val+15;
        b_dim = a_dim;     
\};
\end{alltt}





The syntax is:
\begin{alltt}
<FUNCTION_DEFINITION>  = function ( <NAME_LIST> ) -> <NAME> \{ <EDEN_CODE> \}
\end{alltt}


\section{Linking with eden}
In the previous section the \texttt{eden\{\}} expression was documented. This 
construction provides 
a way of linking a value with an external eden observable.

As well as this feature, two other eden procedures are also provided to assist 
with linking the two notations. The first \texttt{PHYSICAL\_VAL} 
allows an eden observable to be 
dependant on a \%physical observable, and the second, \texttt{physical} 
provides a mechanism to make some \%physical definitions.

Neither of these procedures do anything particularly technical, 
the first liberates the user from having to remember how a \%physical observable
is represented in eden, and the second is just running an execute command.
E.g.

\begin{alltt}
%physical
a = 12 m;
%eden
PHYSICAL_VAL("myObservable","a");
## The current value of myObservable is now [12,"m"]

PHYSICAL("a = 12 kgm\{2\};");
## The current value of myObservable is now [12,"m\{2\}kg"]

\end{alltt}

An improved syntax for linking eden observables to \%physical is discussed in 
section \ref{unitCasting}.


\section{Adding new dimensions and units \label{addingUnits}}

It is desirable that users of \%physical should be able to add their own units 
to the system. Such a feature would enable automatic conversion between units.
Adding extra dimensions is more difficult but could be implemented by adding an 
extra number to the
tail of each dimension vector.

Adding units causes dynamic changes to the parser definitions -- this has been 
implemented with dependency (see section
\ref{grammar}). Providing that the units added do not duplicate the existing 
units this should 
not be a problem. The existing parser needed \texttt{mol} to be searched before
\texttt{m} because of the common prefix between them. As long as the base units 
are matched last then there should be
no common-prefix problems. This assumes that the first characters of 
\texttt{kg} or \texttt{cd} are not used as user-defined units; but since these 
symbols are standard representations for kilo- and
centi- it is probably a fair assumption. 

Parsing might be difficult with many user defined units. For example it would 
be wrong to allow
the definition of hour and hours, because there would be no way of 
distinguishing between hours and hour-seconds. One technique could be to add an 
optional period between each unit. 

\begin{alltt}
%physical
a = 12 m.s{-1};
b = 1 W.hour;
\end{alltt}




\subsection{Methods of defining new units}


Currently, the only way to add a user-defined unit to the notation is by adding 
an entry to a list named
\texttt{user-defined}. This list has an element corresponding to each 
user-defined unit. Provision will have to
be added to the notation to allow new units to be defined from within the 
notation. One such syntax could be:

\begin{alltt}
New unit N = kgms\{-2\} ## the Newton
\end{alltt}

Another idea might be to support full names as well as symbols.

\begin{alltt}
New unit hour (hr) = 3600 s;     ## Add hour unit - can use hour or hr
New unit Newton (N) = kgms\{-2\};  ## Add Newton - can use Newton or N
a = 12 hr;
b = 12 N.hour;
c = 123 Nhr;
\end{alltt}

\subsection{Issues relating to user-defined units}

To be in harmony with the EM paradigm, it should be possible to re-define new 
units during run-time.
For example, initially we may want to declare some imperial units to work in 
our model.

\begin{alltt}
new unit ft = 0.3 m;
...
a = 10 ft; ## internally 3 metres
\end{alltt}

But later we may want to improve our conversion factor:
\begin{alltt}
new unit ft = 0.3048 m;
...
b = 10 ft; ## internally 3.048 metres
\end{alltt}

When we re-defined the foot we changed the parser so that input data could be 
converted into a metric internal representation. Because we are doing this 
conversion in the parsing the result of the above script is that a does not
equal b (even though they have the same definition). A priority for a future 
implementation of \%physical would be
to represent 10 ft as some kind of dependency that is updated when the 
conversion factor changes.

The same problem would arise if I were to decide that I wanted feet to be a 
unit of time. Would it be desirable for this to be allowed to propagate through 
the system?

\section{Changing the grammar of a language as it is parsed \label{grammar}}

At the moment, 
the first unit rule is called by the literal rule. When a new unit is defined,
 a rule between these two that will match the new unit has to be created. 
However, this 
will be problematic if it is desired to \emph{forget} or remove that
 new unit. One solution is to have some kind of function that 
generates the whole parser based on a set of simple definitions that when 
changed cause the whole structure to be rebuilt. Alternatively, the parser 
could make use of the  dependency maintainer and have
a structure where different rules are added using dependency.


For example, consider the following fragment of the \%physical parser:
\begin{alltt}
%eden
PHparse_literal = [ "pivot", " ", ["PHparse_value", "{\bf{PHparse\_unit1}}"],
			["fail", "PHparse_edenObservable"],
			["action", 
			  ["later",
				"$v=[str($p1),str(PH_determineDimension($p2))];"
			  	]]];
...
\end{alltt}
This is a rule that \emph{pivots} on the white space between the value and unit 
of a literal. The value
is then matched by \texttt{PHparse\_value} and the unit by 
\texttt{PHparse\_unit1} shown in bold. We can use
dependency to create a dynamic grammar. For example, the fragment below has had 
\texttt{PHparse\_unit1}
replaced with a different observable that links to the old rule by dependency:
\begin{alltt}
%eden		
PHparse_literal = [ "pivot", " ", ["PHparse_value", "{\bf{PHparse\_unit}}"],
			["fail", "PHparse_edenObservable"],
			["action", 
			  ["later",
				"$v=[str($p1),str(PH_determineDimension($p2))];"
			  	]]];
...
PHparse_unit is PHparse_unit1;
\end{alltt}

The new definitions have not changed the parser; however, we can now introduce 
more dynamic behaviour.
If we have a list called \texttt{user-defined} which has an element for each 
user-defined unit then we can change the
parser based on the existence of extra rules.

\begin{alltt}
PHparse_unit is (user-defined == []) ? PHparse_unit1 : PHparse_newUnit1;
\end{alltt}

As long as any new rules are all linked to \texttt{PHparse\_newUnit1} and 
eventually \emph{fail} back into
the rules for the original units (\texttt{PHparse\_unit1}) then the parser will 
continue to work.





Once we have a dynamic grammar, do we want changes to be made through 
dependency against previous values?



\section{Applying \%physical to the cruise contol model \label{vccs}}

The VCCS is a classic EM model by Ian Bridge \cite{EM_vccs} that models a car 
going over a hill and a cruise control attempting to maintain a constant 
velocity. The model has a number of physical quantities and it is hoped that 
\%physical could help provide an account of the relationships between different 
observables.

There is an LSD account of this model that makes use of units. Bridge's 
implementation refers to this account in the comments and structure of the 
model. His comments provide evidence of
a modelling that makes rich use of dimensional analysis. This is clearly seen in 
the following fragment
of eden definitions.

\begin{alltt}
  pi     = 3.14159;   
  mass   = 2500.0;  /* total mass of car & contents [kg]     */
  windK  = 5.0;     /* wind resistance factor [N m^2 s^2]    */
  rollK  = 50.0;    /* rolling resistance factor [N m^-1 s]  */
  gravK  = 9.81;    /* acceleration due to gravity [m s^-2]  */
  brakK  = 1500.0;  /* braking (viscous) constant [N m^-1 s] */
  forcK  = 40.0;    /* torque to force conversion [m^-1]     */
  sticK  = 100.0;   /* static friction force [N]             */
     \hfill\cite[\texttt{vehicle_dynamics.e}]{EM_vccs}
\end{alltt}

The eden code above could be re-written as:

\begin{alltt}
	%physical
	new unit N = kgms{-2} ## Force = mass x acceleration
                           ## acceleration = distance / time*time
	mass  = 2500.0 kg;
	windK = 5.0 Nm{2}s{2};
	...
\end{alltt}

In the current implementation of \%physical, re-definition of the Newton (line 
1 above) would not propagate through the system (see section 
\ref{addingUnits}). Also the \texttt{new unit} construction has not been added 
to the parser.

Elsewhere in the model some quantities are given in miles per hour and 
\%physical could be used to handle to equivilence
between different units of measure. The conversion between units is currently 
handled by eden functions:

\begin{alltt}
%eden
func mph_to_mps \{
  /* convert miles/hour to metres/sec */
  para mph;
  return 0.448 * mph;
\}

func mps_to_mph \{
  /* convert metres/sec to miles/hour */
  para mps;
  return 2.232 * mps;
\}
\end{alltt}

There should be provision for this kind of task in the 
\%physical notation. This would require some form of `casting' of units
into the desired format.

\section{Managing different output units \label{unitCasting}}

Support is already in \%physical to allow definitions to be made in different 
units. However,
following the implementation of [] all quantities are represented internally in 
terms of base units.


\begin{alltt}
%physical
new unit miles = 1600 m;   ## Define miles
new unit hours = 3600 s;   ## Define hours
mph = 3 milehour\{-1\};      ## %physical stores the value: mph = 16 ms{-1}
\end{alltt}

What is needed is to be able to `cast' into a particular unit for output formats. 
For example,
in the VCCS speeds have to be converted to miles per hour so that the 
speedometer can be defined in terms of miles per hour. We would want to do 
something of the following kind.

\begin{alltt}
%eden
/* Some code defining a speedometer in terms of an observable speedVal */
%eden
PHYSICAL_VAL("speedVal","mph(speed)")
\end{alltt}

Here mph could be defined as a physical function:

\begin{alltt}
mph = function (a) -> b \{
    b_dim = a_dim;
    b_val = scaling_factor * a_val;
\}
\end{alltt}

The difficulty is that this would scale the value but still give the unit ``m'' 
because we are not distinguishing between m and length. This is not ideal. 

One way of solving the problem would be to provide a means for defining a 
dependency between a dimensionless eden observable and a \%physical observable. 
The proposed syntax takes the following form:

\begin{alltt}
%physical 
new unit mile = 1600 m;             ## Definition of mile
new unit hour = 3600 s;             ## Definition of hour

a = 5 ms\{-1\};                       ## a is 5 metres per second

?a;                                 ## Will output current value of a in base 
units

eden\{myVal\} = a;                    ## A new component of %physical that is 
equivalent to making 
                                    ##   the eden definition 
PHYSICAL_VAL("myVal","a");

?a : m->mile,s->hour                ## Will output current value of a in mph

eden\{myval\} =  a : m->mile,s->hour; ## Will make the value of myVal dependent 
                                    ##   on the value of a (in mph)

\end{alltt}

\section{Reflection}
Working with EM has emphasised my need to link computing with the empirical 
sciences. In EM there has been a tendency to withdraw from the formal camps of 
computer science. However, my personal view is that EM is not simply an 
`informal' computer science. Perhaps it is more that those involved in the 
research have found that the formality promoted actually gets in the way of the 
experimental approach to modelling. Taking metrology seriously in our 
programming will improve our ability to add the discipline that those 
interested in formal methods may feel our modelling lacks.

Replacing types with units will provide a discipline of attaching semantics to 
our calculations. In the example of the non-homogeneous equation, that equation 
was lacking a formal semantics and was effectively informality under a formal 
mask. 

In EM, many of our models represent state as experienced in the physical world. 
This is where the power of 
\%physical can be explored. Adding units to the observables can help the 
newcomer to a model get far more meaning from observables than they could 
from raw numbers.


\bibliography{../../sources/metrology,../../sources/modelling}
\bibliographystyle{alpha}

\end{document}
