\documentclass[a4paper]{article}

\title{Dimensionality in Programming}

\begin{document}

\section{Dimensions and programming}

The idea of using dimension as a error-checking mechanism for programming 
has been suggested by a number of people during the last forty years. 
Most writing on the subject follows the lead of \cite{ME_karrLoveman} and 
claims that the earliest mention of the idea was \cite{ME_cheatham}. 
The claim appears to be that dimension should provide an augmentation to the 
existing type-systems and that such provision would provide an extra layer of 
error-protection that a compiler could in theory be built to check.

\cite{grosu} and \cite{?} discuss classic software problems in safety critical
systems caused by confusion about whether values were imperial or metric
and argue that dimensionality in the programming language would have had caught
such errors. It is for these reasons that \cite{?} rejects the idea of run-time
checking of the consistancy of dimension.

Although they might have use in formal verification of software; I would like
to draw attention to a possible use in making source code more readable. This 
use is also pointed out in \cite{ME_karrLoveman} who recognise that attaching
units to variables can provide ``increased readablity''. 

\section{Dimensionality in observables in EM} 

I think that the tkeden tool would be an ideal environment
 to explore the power of programming with dimension. Firstly, the discipline 
of Empirical Modelling makes great use of observation as a concept. 
This leads to 
discussions of a model's observables rather than a program's variables. 

Observation relates the visible state of a model with experience. By this we
mean that observables convey a meaning richer than a value -- which can be
the result of an unfinished computation. 
Observables directly relate to the referent of a model. That is, observables 
in a model will have a context that is more than just a value. Dimension units 
might be found to provide some of that context and help enhance tkeden's 
support for observable-oriented modelling.



\bibliography{../../sources/metrology/metrology}
\bibliographystyle{alpha}

\end{document}
