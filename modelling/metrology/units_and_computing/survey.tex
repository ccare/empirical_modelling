\documentclass[a4paper]{article}

\usepackage{mathrsfs}
\usepackage{anysize}
\marginsize{3cm}{3cm}{2.5cm}{2.5cm}


\newsavebox{\myquoteEnd}
\newenvironment{myquote}[1]
{\begin{sloppypar}\begin{quote}\small\sbox{\myquoteEnd}{#1}\itshape}
{ \ \\ \hspace*{\fill} \usebox{\myquoteEnd} \end{quote}\end{sloppypar}}


\title{Dimensionality in Programming -- Literature summary}
\author{Charles Care}
\date{January 2005}

\begin{document}

\maketitle

\section{Strong Typing and Physical Units \cite{ME_manner}}

This paper draws on the importance of syntax contraints in programming 
language design. M\"anner draws attention to the fact that although strong 
typing is a powerful way of ``forcing compilation checks'' it is ``not as
powerful as the [method] used in all physical calculations'' where quantities 
are associated with units.

M\"anner also draws attention to the distinction between physical dimension 
and physical units. That is that a particular physical quantitiy always has a 
unique dimension, but can be represented by a number of units.
For example, a distance has dimension \emph{length} but can be expressed 
as a quantity with various \emph{units} (e.g. 1 mile, 1.609 kilometers or 
1609 metres.)

The relationship between dimension and typing is then discussed.
The author shows that it is possible to define types for the different 
dimensions required. The following example is given:

\begin{verbatim}
TYPE LENGTH = INTEGER;
     TIME   = INTEGER;
\end{verbatim}

All this really does is to provide `friendly' names for two partitions of the
\texttt{INTEGER} class. Quantities defined using such a type system can be 
treated in two ways. Firstly (M\"anner claims that most PASCAL compilers 
do this), the two types are treated as compatable (since they both have 
the same underlying type). If they are treated as compatable then the compiler
provides no protection against a length being added to a time (which is 
physically meaningless). Secondly, a compiler might treat the two types as
incompatable (M\"anner claims this is what would happen in Ada). As 
incompatible quantities there is protection against length being added to
time; however, multiplication and division are also not allowed (because the 
types are incompatible) and this is a problem because quantities of 
different dimension can be combined in this way.

The implementation suggested by M\"anner is an extension to PASCAL and allows
numeric literals to be given a unit. Units are defined by the programmer 
using the keyword UNIT. 

\begin{verbatim}
UNIT g;        (* mass    *)
     cm;       (* length  *)
     sec;      (* time    *)
     cents;    (* money   *)
\end{verbatim}

From defined units it is also possible to declare a unit as a derivation.

\begin{verbatim}
UNIT k  = 1000;                  (* scale factor    *)
     kg = k * g;                 (* mass            *)
     m  = 100 * cm;              (* length          *)
     N  = kg * m/(sec * sec);    (* force           *)
\end{verbatim}

It is interesting how M\"anner defines the kilogram in terms 
of \texttt{k} and \texttt{g}. It would be a more elegant system if the 
meaning of the SI prefixes (i.e. m-, c-, k-, G- etc.) were defined 
independent of any particular dimension. Otherwise a separate definition will 
be required for each type of unit, adding unnecessary code.

Although there is no discussion about polymorphic types the proposed system
does appear to allow numeric types to be mixed in the conventional ways
of imperitive programming. In the example program \emph{Energy Loss} there is
the statement:
\begin{verbatim}
I := 11.5 eV * ChargeTarget;
\end{verbatim}
where \texttt{ChargeTarget} is a dimensionless \emph{integer} and 11.5 eV has
type \emph{real}. This illustrates an important difference between \emph{type}
which is associated with machine representation and \emph{dimension} which is 
associated with meaning. A quantity's dimension is independent of it's machine 
representation. Type checking prevents run-time problems assocated with the
way the computer processes information, but dimension checking will help
protect against the computer performing calculations that are not meaningful.

\section{Implementation of Physical Units \cite{ME_baldwin}}

This is an account of a simple implementation of physical units as a 
``minimal extension of a commerical Pascal compiler'' and was produced
for a commercial application. The implementation does not provide for user 
defined types and actually was only required to support the five base 
dimensions of 
Charge\footnote{In the SI System, Current is given as a base dimension instead of
Charge. In the SI system Charge would have the base unit $As$
as defined by: $Charge = Current \times Time$
In this system, Charge is a base dimension and so Current would have the base unit
$Cs^{-1}$}
, Length, Mass, Temperature, and Time. These are essentially the base 
dimensions of the SI system for physical units but with the Mole and the Candela
ommitted.

The dimension of a value is represented by an array of five integers. 
Any dimension supported by the system can be written in terms of the five
base dimension and these numbers correspond to the various powers 
of these. For example, velocity is represented $ms^{-1}$ (or in terms of all the
system's base dimensions $C^0m^1kg^0K^0s^{-1}$) this is then represented by 
the computer as \texttt{[0,1,0,0,-1]}.

Baldwin uses the idea of using normalised values and attributes this to 
\cite{ME_physical}. This means that all values of a quantity will be 
represented by the system in terms of its base unit. (E.g. all values of 
electrical current are ``normalized to Ampere''.)

For input values are allowed to have scaling prefixes on their units. 
These are implemented by Baldwin in the lexer of the system. For example $1.2mV$
is translated into $1.2\times10^{-3} V$.

\section{Programming Pascal with Physical Units \cite{ME_physical}}
The authors claim that programming with units can improve ``readability and 
reliabity of software''.

The paper gives a list of necessary ad desirable features of a language that 
incorporates phyical units. 
Attention is drawn to units being a higher level than types and are referred to 
as a ``data attribute'' units can be assigned to constants, types and variables. 
Although dimension is seen as something that could be attached to both integers 
and reals, it is suggested that for simplicity they could be attached just to 
reals.
The proposal is to use the SI standard base units and to reserve the  standard 
prefixes so that defining the meaning of ``kilo'' automatically gives meaning to
km, kg, kA etc. This is their main disagreement frrom manner.

The physical language allows the introduction of  new base units and the 
derivation of units. Also provided is a mechanism for restricting the way that 
a unit can be used. A unit is given one of the following three `sorts'.

\begin{description}
\item[Short] No scaling is allowed (e.g. you dont have milli-minutes)
\item[Single] Can not be used in further derivations
\item[Long] No restriction
\end{description}

A detailed account is given of how WRITE statements are used to deliver units 
with various scaling and precion with or wothout their units.

Physical was written as a preprocessor not a compiler.
a compiler would allow run-time checks and allow input and output routines to 
be integrated.
Future work is outined in three areas:
\begin{enumerate}
\item To produce a complete compiler
\item To manage units with arbitary fixed points (e.g. Celcius and Fareneheit) 
and logarithic units (e.g. the dB). Although it is acknowledged that there is 
probably limited use for such an extension the writers feel that ``allowing 
arbitary and not only multiplicative combinations between units could be an 
interesting extension.''
\item To investigate whether such a system could perform error correction.
\end{enumerate}

\section{A Compact Representation of Units \cite{ME_hamilton}}
A summary of a type calculus is given.

Based on arrays of exponent -- references work by Novak on GLISP.
Although \cite{ME_baldwin} is referenced by Hamilton as an example of
a Pascal extension, no direct reference is made to the fact that the two 
implementations are practically identical. Hamilton differs from Baldwin in 
that the implementation is designed to be compact and so there are heavier
restrictions on the range of values the various element of the array can take.

Emphasis is on the representstion being compact and unambiguous. The system is 
based heavilly on the SI system wih te inroduction of SI units. Extensibility 
is provided withing the SI unit space only. Money has the same dimension as 
number of sugars and rate of change of population is measured in Hz.

More importantly the system doesn't allow for scae factors which makes it 
difficult to support applications where lengths for instance are given in both 
feet and metres. Other writers have seen the protection such a construction 
offers as central to their motivations. However the emphasis on communication 
means that everything is converted back to SI base units before tranmission.

\section{Incorporation of Units into Programming Languages 
\cite{ME_karrLoveman}}

The motivation for this work was the ``design of a language for 
computer-controlled test and diagostic equipment'' where having notations
for physical units would be useful.

Karr and Loveman refer to the ATLAS test language which provided some support
for expressing physical units. They also point to a paper by Cheatham as the 
``earliest mention of the idea''. 

In an attempt to capture what units actualy are, the writers see units as an
``aspect of mode''. Here the real numbers are partitioned by their dimension.

One issue drawn attention to is that although an equation might 
\emph{commensurate} -- that is have the same dimension on both sides. Its units
might not be the same. Karr and Loveman have a vision that the system could 
provide necessary scaling so that 1km + 1mile is a valid formula.



Finally the authors point out the units that would be difficult to apply
to their proposal. These are Celsius and Fahrenheit as well as AD for years.
Although support for these is not generally required in engineering 
computing, it is pointed out that ``they and other even more pathological 
`units', such as the \emph{decibel}, can be elegently handled by the 
`pouches' mechanism''. Pouches are a programming language construct invented by
Cleaveland and are also mentioned by \cite{ME_trg}.


GO THROUGH THE VARIOUS STRENGTHS AND WEAKNESSES HE DESCRIBES AND RELATE THEM TO 
OTHER WRITING.


\section{Rule-based Analysis of Dimensional Safety \cite{ME_grosu}}

Discusses two examples of NASA failures due to quantities being provided in 
the wrong units. It is claimed that dimensional analysis tools would have 
avoided these failures.

The writers mention a C++ library developed for NASA that incorporates
``hundreds of classes representing typical units ... together with appropriate
methods to replace the arithmetic operators''. They are critical of such a 
system and feel that the library ``limit[s] the class of allowable... programs 
to an unnecceptably low level.''

The feeling of the authors is that the concepts of programming with units hasn't
been accepted by ``mainstream programmers'' and that this is because all of the
extensions proposed so far not very light-weight.

The system described in the paper is an augmentation of BC -- a GNU calculator
language. It uses Maude?

\section{Using Units of Measurement in Formal Specification \cite{ME_hayes}}

\section{Units of Measure as a Data Attribute \cite{ME_gehani}}

Gehani proposes an extension to Pascal that supports units called PASCAL/U.
Each expression valid in the language will have two attribute -- a \emph{type}
and its \emph{units}.

\subsection{Unit attributes of expressions}

Gehani's propsals for representation of dimension are similar to those of
\cite{ME_baldwin,ME_hamilton} where a dimension is given by specifying the
composite powers of bas dimension. The units attribute of an expression is 
define as a set of $m$  ordered pairs, where $m$ is a the number of 
base dimensions defined in the system. Each ordered pair contains an identifier
of a base dimension and its corresponding power.

In Gehani's notation
$\mathscr{A}(e)\cdot UNITS = [\langle d_1,x_1 \rangle,
				\langle d_2,x_2 \rangle,...,
				\langle d_m,x_m \rangle ]$
corresponds to the expression $e$ having units $d_1^{x_1}d_2^{x_2}...d_m^{x_m}$

One notable difference between Gehani's notation and other notations is that 
zero-th power dimensions are left out. e.g.
$\mathscr{A}(sales)\cdot UNITS = [\langle dollar, 1 \rangle]$
does not try to list all the other $m-1$ dimensions with a power of zero.

Using this notation, Gehani defines what are legal combinations of various 
units in arithmetical operations.

\subsection{Relating the theory to the PASCAL language}
Gehani decides not to incorporate the UNITS attribute into the type attribute 
because of three main reasons.

\begin{enumerate}
\item Units of measure reflect physical characteristics while types reflect 
means of storage.
\item Different types require different amounts machine storage whereas 
different units do not.
\item Operand types affect the code generated whereas units just show whether 
an operation is legal (although units might provide some scaling factors).
\end{enumerate}

Three representations are considered, but
The syntax chosen is based on the ordered-pairs notation.
Only non-zero dimensions need to be explicitly defined, and the syntax 
allows the exponent of a dimension to be omitted if it is 1.

\begin{verbatim}
const LIGHTSPEED = 3.0E8 UNITS (CM, SEC = -1);
var ROCKETSPEED: real UNITS (CM, SEC = -1);
\end{verbatim}

Gehani notes that it might be desirable to have an equivilence between `feet' 
and `foot' in a dimensional programming language PASCAL/U doesn't support this.

Although PASCAL doesn't support dynamic type changing because of improved
performance and compile-time checks, Gehani decides that units should be
dynamically modifable. This provides the ability to have temporary
variables that can hold data of any dimension. 

\begin{verbatim}
var T: real UNITS(*);
\end{verbatim}

This links with Kennedy's work on dimension and polymorphism \cite{ME_kennedy}.
The following procedure is Gehani's version of dimension polymorphism.

\begin{verbatim}
procedure SWAP (var X, Y : real UNITS(*));
	var T: real UNITS(*);
	begin 	T := X; 
		X := Y; 
		Y := T	end
\end{verbatim}

It is important to note that although the definition of \texttt{SWAP} would
allow the procedure to be called with arguments with different units, such a 
computation would not be allowed by the interal computation where \texttt{X}
 and \texttt{Y} need to have the same dimension.

It is however, possible to set the units of a variable based on the units of 
another.

\begin{verbatim}
var T : integer UNITSOF (A);
\end{verbatim}

Numeric literals have no dimension so that 
\begin{verbatim}
PROFIT := SALES - 1000;
\end{verbatim}
Must be written as
\begin{verbatim}
PROFIT := SALES - 1000 UNITS(DOLLARS)
\end{verbatim}
or
\begin{verbatim}
PROFIT := SALES - 1000 UNITSOF(SALES)
\end{verbatim}

\subsection{Units and dimension}
Gehani's PASCAL/U does ``unitwise'' checking and not dimensional analysis

Hence the following is not allowed.
\begin{verbatim}
var M : real UNITS(MILES);
    K : real UNITS(KILOMETERS);

M=K;
\end{verbatim}

However, Gehani does suggest that conversion could be done with a conversion
factor that had the unit $km\cdot miles^{-1}$ e.g.
\begin{verbatim}
K = 1.6 UNITS(KILOMETERS,MILES = -1) * M;
\end{verbatim}

The syntax of this is improved using the \texttt{counits} keyword.

\begin{verbatim}
UNITS(MILES) = 1.6 UNITS (KILOMETER);
\end{verbatim}







\section{A Proposal for an Extended Form of Type Checking of Expressions 
\cite{ME_house}}

Starts with a discussion about the problems with using types to represent 
units and then summarises Gehani's proposals for unit checking 
\cite{ME_gehani}. In these 
proposals Gehani separates the notions of storage type and units.

However, House raises a number of issues with Gehani's PASCAL/U.
One problem appears to be that in PASCAL/U ``all names of units are 
essentially unrelated.'' There is no way of defining a new unit name -- e.g.
speed. 

For House,  the inclusion of the \texttt{UNIT(*)} construction 
leads to an unnacceptable restriction on the amount of compilation checks 
available.

House then proposes his implementation. He draws a disctinction between 
`dimension' and `unit' and makes use of the SI system with base and derived 
units. It is pointed out that there is no need to include both dimension and
units in the language.
\begin{verbatim}
dim
  metre, kg, sec;
  volume:metre**3;
  unitspeed:metre/sec;
  newton:metre*kg*sec**(-2);

...

type
  speed = real dim unitspeed;
  time = real dim sec;
 
\end{verbatim}

\section{Software support for physical quantities \cite{ME_hall}}
This paper wishes to push the support for physical quantities in programs
further than previous literature. Hall emphasises that ``units and 
measurment scales are closely related.''

\section{Meaning and Syntactic Redundancy \cite{ME_cleaveland}}

Cleaveland introduces the concept of pouches to provide subsets of existing
types. The examples of pouches given show how they can be used to manage 
dimension.
\begin{verbatim}
POUCH years INT age, birth, death;
POUCH people INT population,workers;
...
population := workers + 5 people;
\end{verbatim}

This is effectively a way of using a type system to restrict cross-dimensional
addition. However, Cleaveland does take it further to show how pouches could
model multiplicative combinations of dimensions.

\begin{myquote}{\cite{ME_cleaveland}}
Many users will probably want to say more by saying what kind of relationships 
exist between pouches. For example a variable in pouch x when mulitplies by a 
variable in pouch y produces a variable in pouch z, such as in the case of the 
formula ``pay = rate * time''.
\end{myquote}

One difficulty with the system proposed is that there is no way of giving a 
pouch a name other than its definition. Therefore write the quanity 
$1.5ms^{-1}$ will require the input\footnote{Cleaveand doesn't give us an ascii representation of his 
Algol 68-style syntax but leaves these dots and superscripts.}:
\texttt{1.5 meters\textbullet second\raisebox{1ex}{\small -1}}

Pouches are more general than just dimension and have their own algebra which
forms an Abelian group. Dimensionality is therefore just one instance of how 
pouches could be used in a programmind language.

Cleaveland suggests an implementation where each element of the group can 
be represented by an n-tuple (where n is the number of base pouches) of 
integers. Each integer corresponds to the exponent of the base pouch. 
Although it's not referred to in the paper, this could be extended to a 
tuple of reals should non-integer exponents be required. The problem with this
representation is that the number of base pouches need to be determined to 
decide how these tuples should be. Cleaveland provides no suggestion how this 
might be done. Also it is unclear how the user would be restricted from creating
a new base pouch whose tuple is not linearly-independent from the tuples of the
existing base pouches.

Cleaveland points out that ``generic routines cannot be written for integer 
for any pouch , since the modes of formal and actual parameters must agree.''
This difficulty is got around by introducing the ``hidden pouch'' that
``hides the pouch of the actual operator''. In this provision, Cleaveland's
work is not greatly different from the \texttt{UNITS(*)} construct in
 \cite{ME_gehani}, which was criticised by \cite{ME_house}.





\section{side issue}
Why not have different unit spaces
i.e 
dimension money = SI[0,0,0,0,0,0,0]
10 U[money] + 13 U[money] + money<13 SI[0,0,0,0,0,0,0]>

dimensionless quantiites might be added to others using the ? unit. this allows 
literals to be used for incrementing e.g. in solving an equation for lengths 1m 
to 10m one would want to do 

\begin{verbatim}
for(x=1 m; x <= 10 m; x = x + 1 ?) {...}
\end{verbatim}



WHAT IS SUCH A SYSTEM TRYING TO ACHEIVE?


\bibliography{../../sources/metrology/metrology}
\bibliographystyle{alpha}

\end{document}
